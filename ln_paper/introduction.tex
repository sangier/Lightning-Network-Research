In a world where the use of cash is in its down, we recently learned \cite{1} (some of us with despair) that MasterCard and Google came to a business agreement through which Google would have access to information about millions of transactions performed by Mastercard's customers. Using this information, Google can confirm which adds resulted in actual purchases from the customers. This information is in turn exposed to Google's advertisers, allowing Google to provide extra added value for its ad services. Unfortunately, this is more of a trend than an isolated event. Similar news involving information exchange agreements between other financial companies and other large Internet companies have been published recently as well \cite{2}.
With cash use being drastically limited by practical reasons such as the emergence of e-commerce and by regulatory issues in some countries (e.g., to prevent money laundering \cite{3}) customers concerned with privacy struggle to find an alternative payment method that satisfy their needs.

Crypto-currencies, such as Bitcoin, Ether and others, seem like a promising technology to restore  privacy features in payment operations. Unfortunately, as of today, cryptocurrencies exhibit a number of limitations. A major limitation of current blockchain-based currencies is their scalability. Consider the case of Bitcoin. The number of transactions per second (TPS) that the Bitcoin network can currently process is less than 10. The Visa network is capable of processing over $50,000$ TPS \cite{4} and growing. The limited number of TPS of the Bitcoin network also results in a high transaction fee and high transaction confirmation time, rendering Bitcoin unsuitable for small, day-to-day transactions. In order to become a realistic payment alternative to credit cards, Bitcoin and other cryptocurrencies must reach similar transaction throughput and fees than credit cards.

A promising approach to improve the scalability of cryptocurrencies is to rely on off-chain transactions. The fundamental limitation that Bitcoin and other blockchain-based currencies have is that all transactions are included in the blockchain. By enabling off-chain transactions, i.e., transactions that are not included in the blockchain, Bitcoin throughput can be boosted. The challenge, of course, is how to preserve the guarantees against double-spending and other attacks that are prevented by publishing the transactions in the blockchain.

Lightning is a network for off-chain payments built on top of Bitcoin. It allows the creation of payment channels between two Bitcoin users, and enables secure payments using those channels. The transactions carried out between the endpoints of the channel are not reflected in the blockchain, only the final balance after the channel is closed.  Lightning enables micropayments between any two users connected to the Lightning Network  (LN) by composing routes between the two users by concatenating existing payment channels. Because of its off-chain nature, Lightning is not subject to the limitations in transaction throughput affecting Bitcoin, and it is potentially capable of supporting the required transaction rate to become a viable alternative to traditional payment methods.

However, as currently defined, Lightning has problems that need to be addressed before it can be considered fully operational. The most notable issue identified in Lightning is the routing problem \cite{5}. In Lightning, users perform strict source routing. This means that if a user A wants to transfer a number of bitcoins (BTC) to a user B, user A must define the sequence of channels in the LN from A to B through which this transfer will be routed. In order to determine which channels to use to form a route between A and B, user A retrieves a map of the LN when it first attaches to it. The problem is how to maintain this network map updated, so it can reflect the current capacity of the channels. Broadcasting changes in channel capacities due to transactions executed exhibits scalability and privacy problems. Keeping all channels private imposes the issuer of a transaction to discover a valid path by explicitly pulling information from its neighbours, which is costly and time consuming. The current approach is that LN nodes only announce the initial capacity of the channel. Further changes in the channel capacity due to executed transactions are kept private. The problem with this strategy is that the rate of transaction success can significantly decrease, since users may decide to route transactions based on stale information. It has been reported \cite{6} that the success rate of Lightning transactions involving amounts larger than 6 US\$ is less than 50\% and it drops to 10\% when the transaction amount raises to 60 US\$.  Routing using stale information certainly contributes to the transaction failure rate. 

In this paper we propose to measure how the network map used by the users to create the routes differs from the real network map. In order to do so, we measure the real capacity of the channels and contrast it with the one announced in the network map. In order to do that, we develop a methodology to measure the capacity of the channels in the LN and apply to execute a number of experiments in both the testnet and the real LN. 

\paragraph{Contributions and findings}
We find that blah blah

\paragraph{Paper structure}
The rest of the paper is structured as follows: blah blah
